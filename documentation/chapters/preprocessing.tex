\section{Preprocessing}

Analyzing the data leads to the conclusion that preprocessing steps have to be done. At first attributes are selected,
 that are expected to yield the most improvement in classification accuracy. Then missing values are replaced by 
 mean/mode values. Data was not preprocessed semantically. We did not feel to be able to judge on that without further 
 background knowledge in the field of medicine.
\subsection{Attributes}

In total 22 attributes were removed for the following reasons:

\begin{enumerate}
\item \textbf{Uniqueness}

The first two attributes are unique IDs for the instances. These will not improve classification accuracy, but they 
could make it worse.

\item \textbf{Missing Values}

Two of the attributes are having a lot of missing values: Patient weight has got 97\% missing, payer code 40\%. On that 
account they were removed.

\item \textbf{Unequal Distribution}

A few attributes distributions are heavily biased towards one attribute value. For these attributes instances are having
the same attribute value 99\% of the time. Needless to say they were removed too.

\item \textbf{Single Attribute}

Two of the attributes only have a single attribute value. Without divergence there is no classification, so these are 
removed.
\end{enumerate}

\subsection{Instances}

During preprocessing no instances were removed. Because the main attributes with missing values were removed, all 
instances retained enough data to be of any value for building a classification model. Less than 1\% of the data was 
missing. These missing values were replaced by means for numeric and modes for nominal data.


\subsection{Conclusion}

Affords were made to semantically split the data and analyze the value of different types of attributes. The removal of 
such groups or single attributes did not lead to an observable improvement in accuracy, therefore all remaining 
attributes were left in the training data.