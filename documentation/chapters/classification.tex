\section{Classification}
\label{classification}

Was istr das problem?
3-class problem


welche vorgaben?
minimum
\begin{itemize}
	\item[\textbullet] algorithm based on SVM
	\item[\textbullet] algorithm for semi-supervised classification
	\item[\textbullet] one additionally supervised algorithm
\end{itemize}


overview of:
used classifier





\begin{itemize}
	\item[\textbullet] NaiveBayes
	\item[\textbullet] DecisionTable
	\item[\textbullet] LibSVM
	\item[\textbullet] YATSI
\end{itemize}


zur eveluirung der trainierten classifizieren verwendeten wir die beiden üblichen evaluirungs methoden:
\textbf{Percentage Split} - used with 80\% as training set and 20\% as test set\\
\textbf{Cross-Validation} - used with 10 folds\\


während der durchsicht der ergebniise ist uns aufgefallen dass die ergebisse bei vielen classificatoren alle instanzen in eine classe classifizierten. das problem entsteht daraus dass die anzahl der insanzen zwischen der classen stark variiert.
UM dieses problem zu reduzierten haben wir zwei ansätzen ausprobiert die in section \nameref{ocp}
vorgestellt werden.





\begin{table}[h]
\tbl{Classifier result for each evaluation method}{%
\begin{tabular}{ | l | l | l |}\hline
    Algorithm & Evaluation & Correctly Classified  \\ \hline 
    NaiveBayes\%  & Split\%  & 57.32\% \\ \hline 
    NaiveBayes\%  & Cross-Validation\%  & 57.14\% \\ \hline 
    DecisionTable\%  & Split\%  &  57.53\% \\ \hline 
    DecisionTable\%  & Cross-Validation\%  & 57.75\% \\ \hline 
    SVM\%  & Split\%  & -\% \\ \hline 
    SVM\%  & Cross-Validation\%  & -2\% \\ \hline     
    YATSI\%  & Split\%  & -\% \\ \hline 
    YATSI\%  & Cross-Validation\%  & -\% \\ \hline 
    
\end{tabular}}
\label{table:resultsClassifiers}
\end{table}

results mit alle in 1 klasse

explain the results
