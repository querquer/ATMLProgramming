\section{Classification}
\label{classification}

The attribute to be classified from the data set is \textit{Readmitted} and exist in three characteristics, \textit{No},\textit{ \textless 30} and \textit{ \textgreater 30}. These values denote if the patient was readmitted in more or less than 30 days, and “No” for no record of readmission. \cite{OlmoMedicos}

The distribution of the values in these three instances differs greatly from 54864 instances in \textit{No} and 35545 instances in \textit{ \textless 30} to 11357 instances in \textit{ \textgreater 30}. These unequal distribution led to a problem that is addressed in section \nameref{ocp}. 

When selecting classifiers The following criteria were important: performance, type specification from task description and feasibility.
In addition, many classifiers omitted because they could not solve a 3 classes problem.
\\

Used classification algorithms:
\begin{itemize}
	\item[\textbullet] NaiveBayes, a supervised algorithm 
	\item[\textbullet] DecisionTable, a supervised algorithm 
	\item[\textbullet] LibSVM, a algorithm based on SVM from 
	\item[\textbullet] YATSI, a semi-supervised algorithm
\end{itemize}



For the evaluation of the trained classifiers, we used the two common evaluation methods:\\
\textbf{Percentage Split} - used with a 50\%/50\% split\\
\textbf{Cross-Validation} - used with 10 folds\\



By means of the presented classification and evaluation methods, we achieved the following results:
\begin{table}[h]
\tbl{Classifier result for each evaluation method}{%
\begin{tabular}{ | l | l | l |}\hline
    Algorithm & Evaluation & Correctly Classified  \\ \hline 
    NaiveBayes  & Percentage Split  & 57.32\% \\ \hline 
    NaiveBayes  & Cross-Validation  & 57.14\% \\ \hline 
    DecisionTable  & Cross-Validation  & 57.75\% \\ \hline 
    SVM  & Percentage Split  & 60.4765\% \\ \hline 
    SVM  & Cross-Validation  & 60.4186\% \\ \hline     
    YATSI  & Percentage Split  & 53.22\% \\ \hline 
    YATSI  & Cross-Validation  & 53.30\% \\ \hline 
    
\end{tabular}}
\label{table:resultsClassifiers}
\end{table}
63.3014 %
62.3201 %

explain the results\\
kaum unterschied zwischen den evaluierungs methoden\\



During the review of the results we noticed, that some classification algorithms put all test instances in one class. The problem arises from it that the number of instances between the classes varies greatly. To counteract this problem, we tried two approaches presented in section \nameref{ocp}.

