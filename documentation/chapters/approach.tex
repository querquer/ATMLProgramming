\section{Approach}
\label{approach}


Our general approach to solve the task was as follows:
\\

In our procedure we oriented our working steps on the task description:
\begin{itemize}
	\item[\textbullet] analysed the data set and its various attributes
	\item[\textbullet] cleaned the data (remove attributes, missing values)
	\item[\textbullet] used different libraries which provides different machine learning algorithms
	\item[\textbullet] evaluated the quality of selected algorithms with accuracy metric and cross-validation
\end{itemize}
%In our procedure we oriented our working steps on the task description:
%\begin{itemize}
%	\item[\textbullet] analyse the data set and its various attributes
%	\item[\textbullet] clean the data (e.g. missing values)
%	\item[\textbullet] select an appropriate subset of the attributes and explain your choice
%	\item[\textbullet] use different suitable machine learning algorithms (either implement them, or use existing libraries, e.g. Weka)
%	\item[\textbullet] determine the quality of your model (e.g. through cross-validation, log loss3, confusion matrix)
%	\item[\textbullet] compare your results between different algorithms
%\end{itemize}
%In addition should the classification methods used include the following:
%\begin{itemize}
%	\item[\textbullet] algorithm based on SVM
%	\item[\textbullet] algorithm for semi-supervised classification
%	\item[\textbullet] one additionally supervised algorithm
%\end{itemize}
In order to implement the classification algorithms we used different libraries: \\
\\
\textbf{Weka}\footnote{\url{http://www.cs.waikato.ac.nz/ml/weka/}} -
collection of machine learning algorithms for data mining tasks which provides a lot of basic functionality like data storage, filtering, visualisation, and additionally many classifier and clustering algorithms \\
\\
\textbf{Collective Classification}\footnote{\url{http://www.cms.waikato.ac.nz/~fracpete/projects/collective-classification/}} - 
provides semi-supervised algorithms and extends the Weka library \\
\\
\textbf{LIBSVM}\footnote{\url{http://www.csie.ntu.edu.tw/~cjlin/libsvm/}} - 
is an integrated software for support vector classification and provides a implementation of SVM classifier for Weka \\
\\
For our solution we selected these four classifiers:


In addition should the classification methods used include the following:
\begin{itemize}
	\item[\textbullet] NaiveBayes
	\item[\textbullet] Decision Table
	\item[\textbullet] SVM
	\item[\textbullet] YATSI
\end{itemize}
	\item[\textbullet] algorithm based on SVM
	\item[\textbullet] algorithm for semi-supervised classification
	\item[\textbullet] one additionally supervised algorithm
\end{itemize}

For the development our program we use the following libraries:\\
\textbf{weka} - provides a lot of basic functionality like data storage,filtering, visualisation and additionally many classifier and clustering algorithms \\
\textbf{collective classification} - provides semi-supervised algorithms \\
\textbf{libSVM} - provides additional SVM algorithms \\

