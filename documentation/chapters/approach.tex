\section{Approach}
\label{approach}


an der aufgabenstellung angelehnt

\begin{itemize}
	\item[\textbullet] analyse the data set and its various attributes
	\item[\textbullet] clean the data (e.g. missing values)
	\item[\textbullet] select an appropriate subset of the attributes and explain your choice
	\item[\textbullet] use different suitable machine learning algorithms (either implement them, or use existing libraries, e.g. Weka)
	\item[\textbullet] determine the quality of your model (e.g. through cross-validation, log loss3, confusion matrix)
	\item[\textbullet] compare your results between different algorithms
\end{itemize}

für die bearbeitung dieser aufgaben verwendeten wir folgende libs: 
weka - für data integrate processing and basic classification algo
collective classification - die uns semi-supervisted anbietet
libSVM - die uns zusätzliche SVM algo anbietet

weka ist eine 