\section{One Class Problem}
\label{ocp}

The results some classification algorithms put all test instances in one class. The problem arises from it that the number of instances between the classes varies greatly. To counteract this problem, we tried two approaches presented in the following.

\subsection{Even Classes}
\label{even}
In this approach, we have adjusted the number of instances of each class. We have first determines the class with least number of instances. Then we have the other classes cleared by random selection so much instances until the classes were equalized.

With this newly created data set we have carried out our classifications again.
Surprisingly, now showed the same problem. Except that now the characteristics, \textit{ \textgreater 30}, with the former fewest instances, get all test instances.
Because of these poor results, we did not pursue this approach. This facilitates the task of classification.




\subsection{2 Class Problem}
\label{2class}

In this approach, we converted the 3-class problem to a 2-class problem.
We have the attribute \textit{Readmitted} changed so that
the smaller characteristics, \textit{ \textless 30} and \textit{ \textgreater 30}, were summarized.
That reduced the expressiveness of the attribute. Now you can only make a statement about if the patient was readmitted or not.

With this newly created data set we have carried out our classifications again. We achieved better results than with the original data set. As this is not compatible with the tasks description, we did not pursue this approach.
