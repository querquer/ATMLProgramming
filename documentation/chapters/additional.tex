\section{One Class Problem}
\label{ocp}

During experimental evaluation of different classifiers some classification algorithms assigned the same label to every 
instance. The problem arises from an unequal distribution in the target class. To tackle this problem, we have tried 
two approaches presented in the following.

\subsection{Even Classes}
\label{even}
In this approach, we have adjusted the number of instances for each attribute value in the target class. We have first 
determined the class with the least number of instances. Then we have removed instances from other classes until
an equal distribution was obtained.

With this newly created data set we have carried out our classifications again.
Surprisingly, the results were showing the same problem again. However the classifiers assigned the label 
\textit{ \textgreater 30}, with the former fewest occurrences, to every instance.
Because of these poor results, we did not pursue this approach any further.




\subsection{2 Class Problem}
\label{2class}

In this approach, we converted the 3-class problem to a 2-class problem.
We have got the target attribute \textit{Readmitted} changed in order to summarize the characteristics with fewer 
occurrences, \textit{ \textless 30} and \textit{ \textgreater 30}.
The goal is to reduce the expressiveness of the attribute. With this change the only possible statement about the 
readmission of a patient is whether it took place or not.

On this newly created data set we have carried out our classifications again, yielding better results than using the 
original one. As this is not compatible with the tasks description, we did not pursue this approach.
